\documentclass[a4paper,12pt]{article}
\usepackage[utf8]{inputenc}
\usepackage[brazil]{babel}
\usepackage{lipsum}
\usepackage{graphicx}
\usepackage{hyperref}
\usepackage{amsmath}
\usepackage[left=2cm, right=2cm, top=3cm, bottom=3cm]{geometry}

\title{Título do Artigo}
\author{Seu Nome}
\date{\today}

\begin{document}

\maketitle
\begin{abstract}
\lipsum[1]
\end{abstract}

\newpage

\tableofcontents
\newpage

\section{Introdução}
\lipsum[2-3]

\subsection{Como funciona}
\lipsum[1]

\section{Desenvolvimento}
\lipsum[4-6]

\newpage

\section{Conclusão}
\lipsum[7]
\newpage

\begin{thebibliography}{9}
\bibitem{ref1} Autor, \textit{Título do Livro}, Editora, Ano.
\bibitem{ref2} Autor, \textit{Título do Artigo}, Revista, Volume, Número, Ano.
\end{thebibliography}

\end{document}
