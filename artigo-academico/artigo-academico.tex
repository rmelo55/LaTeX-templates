\documentclass[a4paper,12pt]{article}
\usepackage[utf8]{inputenc} % .. Suporte para caracteres UTF-8
\usepackage[T1]{fontenc} % .. Codificação de fontes para suportar caracteres acentuados
\usepackage[brazil]{babel} % .. Suporte para português brasileiro
\usepackage{amsmath} % .. Pacote para fórmulas matemáticas
\usepackage{amssymb} % .. Símbolos matemáticos adicionais
\usepackage{graphicx} % ........ inclui de gráficos e figuras
\usepackage{indentfirst} % ..... identa o primeiro parágrafo de cada seção
\usepackage[bottom=2cm, top=3cm, left=3cm, right=2cm]{geometry} % ..ajuste das margens
\usepackage{helvet} % .. Uso da fonte Helvetica
\renewcommand{\familydefault}{\sfdefault} % .. Definir fonte sans-serif como padrão
\usepackage{url} % .. Suporte para URLs
\usepackage{lipsum} % .. Gerador de texto exemplo
\usepackage{hyperref} % .. Criação de hyperlinks
\usepackage{tocloft} % .. Personalização do sumário
\usepackage{setspace} % .. Controle do espaçamento entre linhas

\onehalfspacing  % .. espaçamento entre linhas de 1.5

\begin{document}

\begin{center}
    \textbf{\Large Uso de Tecnologias da Informação no Ensino} \\ [3cm]
    % \vspace{3cm}
\end{center}

\begin{flushright}
    Seu Nome \\[0.5cm]
    \today \\
\end{flushright}

\begin{abstract}
\noindent
Este estudo tem como objetivo explorar os mecanismos subjacentes e as implicações de diversos processos dentro de um determinado campo de pesquisa. Através da análise de dados coletados de múltiplas fontes, foi alcançada uma compreensão abrangente sobre o tema. Os métodos empregados incluem abordagens qualitativas e quantitativas, permitindo uma perspectiva equilibrada e objetiva. Os resultados sugerem que vários fatores contribuem significativamente para os desfechos, destacando a complexidade e a inter-relação dessas variáveis. Além disso, os achados indicam potenciais aplicações em campos correlatos, fornecendo uma base para pesquisas futuras. Esta investigação ressalta a importância de estudos contínuos, à medida que novos desenvolvimentos e inovações podem surgir a partir deste trabalho fundamental. As implicações do estudo são amplas, com possíveis efeitos sobre a indústria, a academia e a formulação de políticas. \\ \\
\textbf{Palavras-chave}: palavra1, palavra2, palavra3.
\end{abstract}

\newpage

\renewcommand{\contentsname}{\centerline{\textbf{SUMÁRIO}}}
\tableofcontents
\newpage

\section{Introdução}
\lipsum[3-4] % apague essa linha e coloque o seu texto

\subsection{Como funciona}
\lipsum[5-6] 
\cite{ref1}  % use esse no meio do texo logo após a palavra para citar o autor

\subsection{Características do objeto de estudo}
\lipsum[7-8]
\cite{ref1}

\section{Desenvolvimento}
\lipsum[9-10]
\cite{ref2}

\subsection{Comportamento dos participantes}
\lipsum[11-12]

\subsection{Reação após mudanças nas regras}
\lipsum[13-14]

\newpage

\section{Considerações Finais}
\lipsum[15-16]

\newpage

\begin{thebibliography}{9} % caso você tenha 10 ou mais referências, digite 99
    \bibitem{ref1} Autor1. \textit{Título do Livro}. Editora, Cidade, Ano.
    \bibitem{ref2} Autor2. \textit{Título do Artigo}. Revista, Volume, Número, Ano. Disponível em: \url{https://saraiva.com.br}. Acesso em: 30 jul. 2024.
\end{thebibliography}

\end{document}
